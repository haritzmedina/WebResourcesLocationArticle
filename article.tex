
%Document generated using DScaffolding from https://www.mindmeister.com/1355731819
\documentclass{article}
\usepackage[utf8]{inputenc}

\newcommand{\todo}[1] {\iffalse #1 \fi} %Use \todo{} command for bringing ideas back to the mind map

\title{WebResourcesLocation}
\author{}

\begin{document}

\maketitle
      

\section{Introduction}

%Describe the practice in which the problem addressed appears
Annotation of web resources encompasses different activities: User navigates to the resource that wants to annotate, Opens the resource with(in) the annotation system, User annotates content, User (or another user) searches in annotation system for previously done annotations and User revisits the annotation in the context. As for User navigates to the resource that wants to annotate, it has been described as Redirections may occur during this process. As far as Opens the resource with(in) the annotation system is concerned, it has been described as The annotation system comes in different "flavours" or implementations (as a web holding the annotable document, as a proxy server, as a browser extension. As regards User annotates content, it has been described as Annotation target is composed by an IRI (or group of IRIs), and They can be shared with other people (annotators or not). It is conducted using Annotation system saves the annotation target, which is composed by an IRI (or more than one IRI) and optionally a group of selectors. As for User (or another user) searches in annotation system for previously done annotations, it is conducted using Hypothes.is search engine. 
    
%Describe the practical problem addressed, its significance and its causes
A major problem within Annotation of web resources is that Annotated resources are difficult to identify robustly. This problem is of particular concern as it is now well established that it can lead to Orphaned annotations, unable to relocate or reopen the target where the annotation was done \cite{Aturban2015} \cite{Aturban2015} and Difficulties retrieving annotations for the visited/opened resource \cite{Klein2018}. Causes can be diverse: (1) Digital libraries and file hosting services have a landing page which redirects to a temporal URL to serve web resources, (2) Some digital libraries don't attach metadata to publication resources, (3) Same resource is located in multiple locations, (4) Same resource is represented in different formats \cite{Van de Sompel2016} \cite{Van de Sompel2016} \cite{Van de Sompel2016}, (5) Some hosting services don't allow to open web resources inline, as they are automatically downloaded (Content-Disposition: attached) and (6) Reference rot or links are broken \cite{Klein2014} \cite{Klein2014}. 
    
%Summarise existing research including knowledge gaps and give an account for similar and/or alternative solutions to the problem

    
%Formulate goals and present Kernel theories used as a basis for the artefact design
In this work, we address 1 main cause: Some digital libraries don't attach metadata to publication resources. 
    
%Describe the kind of artefact that is developed or evaluated
This article presents a novel artefact
    
%Formulate research questions

    
%Summarize the contributions and their significance

      
%Overview of the research strategies and methods used
This article has followed a Design Science Research approach.

%Describe the structure of the paper
The remainder of the paper is structured as follows: 

%Optional - illustrate the relevance and significance of the problem with an example
    
      
\bibliographystyle{unsrt}
\bibliography{references}

\end{document}
    